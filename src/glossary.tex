\newglossaryentry{Umgebung}{
  name=Umgebung,
  description={
    In der Softwareentwicklung wird häufig von produktiven Umgebungen
    gesprochen. Damit ist, aus betrieblicher Sicht eine Zusammenstellung aus
    verschiedenen Systemen in definierter Konfiguration gemeint, auf denen die
    entwickelte Software produktiv eingesetzt wird. Um Fehlentwicklungen zu
    vermeiden, wird die Software häufig in sogenannten Entwicklungsumgebungen,
    die der Produktionsumgebung nachempfunden sind getestet \cite[S.~49,
    250]{cd}
  }
}

\newglossaryentry{wrapper}{
  name=Wrapper,
  description={
    Als Wrapper wird eine Software bezeichnet, die eine andere Software
    vollständig oder teilweise umschließt und ggf. um weitere Funktionalitäten
    ergänzt oder verändert
  }
}

\newglossaryentry{OpenSource}{
  name=Open Source,
  description={
    Mit Open Source(OS) Software wird Software bezeichnet, deren Quelltext frei
    und öffentlich verfügbar ist. Ausschlaggebend ist, dass er der
    Allgemeinheit uneingeschränkt zur Verfügung steht, bzw. auf Anfrage zur
    Verfügung gestellt wird. Dieser darf also gelesen, genutzt und verändert
    werden. Die Veränderungen können, dürfen und sollten der Allgemeinheit
    ebenso wieder zur Verfügung gestellt werden. Mehrheitlich kann Open Source
    Software kostenlos genutzt werden. Dieses Vorgehen wird von der Open Source
    Gemeinschaft (Open Source Community) durch allgemein anerkannte Regeln und
    Lizenzmodelle bekräftigt\cite{osi_d,osi_l}}
}

\newglossaryentry{tag}{
  name=Tag,
  description={
    Durch den Benutzer erstellte Markierung für einen bestimmten Stand
    innerhalb eines Repositorys. Wird i.d.R. für Releases oder Versionen
    verwendet und wird i. d. R. nicht weiter verändert. \textbf{Abschnitt
    \ref{sec:tag}}
  }
}

\newglossaryentry{commit}{
  name=Commit,
  description={
    Stellt einen eindeutigen Stand des Repositorys dar. Entält i.d.R.
    Informationen über den Zeitpunkt der Erstellung, Name des Autors und eine
    Bemerkung über die durchgeführten Änderungen. \textbf{Abschnitt
    \ref{sec:commit}}
  }
}

\newglossaryentry{repository}{
  name=Repository,
  description={
    Datenspeicher, in dem Dateien durch das Versionskontrollsystem verwaltet
    werden. \textbf{Abschnitt \ref{sec:repository}}
  }
}

\newglossaryentry{branch}{
  name=Branch,
  description={
    Durch den Benutzer erstellter Abzweig an einer bestimmten Stelle bzw. Stand
    eines Repositorys. \textbf{Abschnitt \ref{sec:branch}}
  }
}

\newglossaryentry{sha1}{
  name=Secure Hash Algorithm,
  description={
    Wird durch \acrshort{sha-1} abgekürzt und ist ein für kryptografische Zwecke
    entwickelter Hash-Algorithmus, der eine Prüfsumme aus digitalen Daten z.B.
    Dateien bildet. Es handelt sich um eine nicht umkehrbare mathematische
    Funktion, die eine Bit-Folge auf eine Prüfsumme von 160 Bit Länge
    abbildet \cite[S.~50]{gitosp}}
}

\newglossaryentry{HEAD}{
  name=HEAD,
  description={
    Eine Referenz auf einen letzten/neuesten Commit eines Branches/Repositorys.
    \textbf{Abschnitt \ref{sec:head}}
  }
}

\newglossaryentry{sshell}{
  name=Secure Shell,
  description={
    Wird durch SSH abgekürzt und ermöglicht eine
    Verbindung zu einem entfernten Rechner mittels des gleichnamigen
    Protokolls
  }
}

\newglossaryentry{repourl}{
  name=Repository-URL,
  description={
    Zieladresse des Repositorys. Enhält Informationen über die Zugriffsmethode,
    Zugriffsparameter und den Ort des Repositorys \cite[S.~141]{gitosp}    
  }
}
