\chapter{\result}\label{cha:result}
Im Rahmen dieser Arbeit wurden grundlegende Funktionen und Arbeitsweisen von
Versionskontrollsystemen im Allgemeinen und Git im Besonderen behandelt.
Die Möglichkeit, alle Dateien dauerhaft in jeder Version zuverlässig und
reproduzierbar zu erhalten, ist ein erheblicher Zugewinn verglichen mit der
Zeit, in der Quellcode ein einsames Dasein auf der Festplatte eines Computers
fristete. Ein wichtiger Punkt ist jedoch nicht nur das leichtere
Verwalten von Dateien, sondern auch der Gedanke, dass überholte Ideen oder
Entwicklungen ohne das Risiko von Verlusten entfernt oder überarbeitet werden
können. Diese Sicherheit schafft Freiheit, neue Dinge auszuprobieren
oder zu verbessern. Das unterstreichen auch die Autoren Jez Humble und David
Farley in \cite[S.~35]{cd} mit dem folgenden Satz:

\begin{center}
\textit{\glqq{}Version Control: The Freedom to Delete}\grqq{}.
\end{center}

Ungeachtet der beschriebenen Einschränkungen ist Git im Vergleich zu
vorhergehenden Versionskontrollsystemen ein leistungsfähiges Werkzeug, welches
nicht nur technische Verbesserungen mitbringt, sondern die Zusammenarbeit in
Teams und Projekten auch äußerst effizient unterstützt. Die Bedienung von
Git mag zu Anfang nicht besonders übersichtlich sein, insbesondere bei
fortgeschrittenen Themen im Rahmen umfangreicher Projekte, ermöglicht die
Architektur von Git aber eine Skalierbarkeit, die mit anderen
Versionskontrollsystemen nur schwer zu erreichen ist.

Diese in LaTex gesetzte Seminararbeit stellt mithilfe des verwendeten Beispiels,
den Einsatz von Git in einem kleinen Rahmen vor und wurde ebenfalls mit Git
versioniert\footnote{https://github.com/sagiru/Seminar-1908-Git/}. Dies zeigt,
dass Projekte mit kleinem Umfang, die nicht in Teams geteilt und bearbeitet
werden, gleichermaßen von der Arbeit mit Git profitieren.

Für eine Anwendung über den Inhalt dieser Seminararbeit hinaus, ist es
erforderlich, Themen wie Bisektion, Submodule, Workflows und die Zusammenarbeit
von Git mit anderen Versionskontrollsystemen zu betrachten. Die intensivere
Auseinandersetzung mit dem Objektmodell führt ebenfalls zu einem besseren
Verständnis der Funktionsweise von Git.

In der Praxis hat der Umgang mit Git gezeigt, dass sich Softwarequalität und
Kollaboration in Teams durch den Einsatz von Plattformen wie
Gerrit\footnote{https://www.gerritcodereview.com/} oder
GitHub\footnote{https://github.com} nochmals verbessern lassen. Die Verwendung
von Versionskontrollsystemen ist eine Voraussetzung sowohl für den Einsatz
agiler Vorgehensmodelle in der Softwareentwicklung, als auch für eine
erfolgreiche Auseinandersetzung mit Themen wie \textit{Continuous Integration}
oder \textit{Continuous Delivery}.
