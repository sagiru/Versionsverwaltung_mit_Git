\chapter{\result}\label{cha:result}
Im Rahmen dieser Arbeit wurde grundlegend über Funktionen und Arbeitsweisen von
Versionskontollsystemen im Allgemeinen und Git im Speziellen gesprochen. Die
angesprochenen Themen wurden meist nur oberflächlich behandelt und
fortgeschrittene Betrachtungen, aus Platzgründen, häufig vollständig
weggelassen. Allein aber die Tatsache, dass alle Dateien über die Zeit in
jeder Version zuverlässig und reproduzierbar erhalten bleiben, ist in
erheblicher Zugewinn gegenüber einer Zeit, in der Quellcode ein einsames Dasein
auf der Festplatte eines einzenlnen Computers frisitete. Jez Humble und David
Farley sprechen darüber hinaus in \cite[S.~35]{cd} von:

\begin{center}
\textit{\glqq{}Version Control: The Freedom do Delete}\grqq{}.
\end{center}

Damit meinen die Autoren nicht das leichtere Verwalten von Dateien, sondern
eher die Idee, dass \textit{alte} Ideen oder Entwicklungen ohne das Risiko von
Verlusten entfernt oder überabreitet werden können. Diese Sicherheit schafft
\textit{Freiheit}, neue Dinge auszuprobieren oder zu verbessern.

Ungeachtet der in Abschnitt \ref{cha:lookout} beschriebenen Einschränkungen, ist
Git im Vergleich zu vorhergehenden Versionskontrollsystemen, ein mächtiges
Werkzeug welches nicht nur technische Verbesserungen mitbringt, sondern die
Zusammenarbeit in Teams und Projekten auch äußerst effizient unterstützen kann.
Die Bedienung von Git, hat man nun Erfahrungen mit anderen
Versionskontollsystemen oder nicht, mag zu Anfang etwas sperrig sein.  Wenn man
sich aber mit fortgeschrittenen Themen im Rahmen umfangreicherer Projekte
auseinandersetzt (z.B. Abschnitt \ref{sec:kernel}), stellt man fest, dass
die Architektur von Git eine Skalierbarkeit, ermöglicht die man mit anderen
Systemen vergeblich sucht. Das in der Arbeit verwendete Beispiel(Abschnitt
\ref{sec:first_commits}) zeigt aber auch die Arbeit mit Git in einem kleinen
Rahmen. So wurde auch diese Seminararbeit, die in LaTeX gesetzt wurde, mit Git
versioniert\footnote{https://github.com/sagiru/Seminar-1908-Git/}. Auch
\textit{kleine} Projekte, die nicht in Teams geteilt und bearbeitet
werden profitieren von der Arbeit mit Git.

In der Praxis hat der Umgang mit Git gezeigt, dass sich Qualität von erstelltem
Quellcode und Kollaboration in Teams durch den Einsatz von Plattformen wie
Gerrit\footnote{https://www.gerritcodereview.com/} oder
GitHub\footnote{https://github.com}, nochmals verbessern lassen. Der Einsatz von
Versionskontrollsystemen ist sowohl Vorrausetzung für eine erfolgreiche
Auseinandersetzung mit Themen wie \textit{Continuous Integration} oder
\textit{Continuous Delivery}, als auch für den Einsatz agiler Vorgehensmodelle
in der Softwareentwicklung.\cite{cd}

Weiterführende Begriffe wie Regression und Bisektion, Stashes, Submodulen,
Workflows und Reviewsystemen, wurden in dieser Arbeit, aus Platzgründen nicht
betrachtet. Auch muss für den Umgang mit Konflikten, die durch die
gleichzeitige Änderung von mehreren Personen an gleichen Dateien verursacht
werden können, auf weitere Fachliteratur verwiesen werden. Die o.a. Themen
werden beispielsweise in \footnote{\cite{gitosp},\cite{progit},\cite{gitwf}
oder \cite{cd}} ausführlich behandelt.

Nicht zuletzt, bietet der Funktionsumfang und die Manpages der Kommandos auch
für Personen mit viel Erfahrung eine umfangreiche Quelle um Herausforderungen,
im Rahmen der Zusammenarbeit von Entwicklerteams, in der Praxis zu lösen.
