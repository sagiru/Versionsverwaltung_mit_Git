\chapter{Versionsverwaltung}\label{cha:Versionsverwaltung}
\section{Definition}\label{sec:Definition}
Versionskontrollsysteme sind auch bekannt als Versionsverwaltungssysteme (engl.
\acrlong{vcs}), Quellcode Verwaltung(engl. Source Control) oder
Revisionskontollsysteme(engl. Revision Control System). Mit diesen Begriffen
sind Systeme gemeint die es Entwicklern, Teams oder Organisationen erlauben
eine vollständige Historie mit allen Änderungen an dem Quellcode ihrer
gemeinsam entwickelten Software zu verwalten. Ausschlaggebend ist hierbei das
für alle Nutzer transparent wird wer, wann und vor allem warum welche
Änderungen durchgeführt hat. Eine weiterer wichtige Eigentschaft ist das es
verschiedenen Teams eine Zusammenarbeit an ggf\. verschiedenen Teilen der
Software ermöglicht ohne sich gegenseitig zu
behindern\footnote{\label{dev:1}Das hängt natürlich nicht nur von dem
Versionskontollsystem ab sondern auch von dem Design der entwickelten Software.
Diese wird i.d.R. eher modular aufgebaut so das die Möglichkeit einer paralelle
Entwicklung unterstützt wird.}.\cite[s.~381]{cd}

\section{Geschichtliche Entwicklung}\label{sec:GeschichtlicheEntwicklung}
Das erste Versionskontrollsystem namens SCCS enstand 1972 und wurde von Marc J.
Rockkind bei Bell Labs
geschrieben\footnote{\url{http://www.belllabs.com/}}\cite[s.~382]. Ab diesem
Zeitpunkt enstand eine Vielzahl von verschiedenen Versionskontollsystemen. Hier
seien nur einige genannt. Als Alternative zu dem properitären SCCS folgte
Anfang 1980 das von Walter F. Tichy an Purdue University entwickelte erste
\acrlong{OpenSource} Versionskontrollsystem
\acrfull{rcs}\cite{paper:rcs,link:rcs}. Ross Ridge schrieb 1993 mit
\acrshort{mysc} einen freien Ersatz für \acrfull{sccs} der in späteren
Versionen in \acrfull{cssc} umbenannt wurde\cite{link:cssc,link:mysc}. Alle
drei Systeme finden in der Praxis nur noch wenig Anwendung und daher wird an
dieser Stelle nicht auf weiter auf Details eingegangen.
\subsection{CSV}
\subsection{SVN}
\subsection{Weitere}
\section{Versionsverwaltungssysteme}
\subsection{Lokal}
Beide Systeme arbeiteten auf dem lokalen
Dateisystem.,
\subsection{Zentral}
\subsection{Dezentral}
\subsection{Streaming}
\section{Warum und Wozu?}
single vs big teams, picture lokal vs team
\label{sec:why}
\section{Konzepte}
\label{sec:systems}
\section{Konzepte}
\label{sec:Konzepte}
\subsection{Datenhaltung}
\label{sec:Datenhaltung}
\subsection{Tags}
\label{sec:Tags}
\subsection{Branches}
\label{sec:Branches}
